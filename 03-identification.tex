% !TEX root = main.tex
\section{Identification}
\label{sect:identification}

Before looking into the developer discussion to explore some design concept of the project, we need to first understand how do the software developers understand and identify the design points. A study by Sousa et al.,\cite{Sousa2017} revealed six preeminent strategies named as: smell-based, problem-based, principle- based, element-based, quality attribute-based and pattern- based that are adopted by developers to identify the design issues. After the strategies are revealed, they experimented with familiarity and unfamiliarity scenario to qualitatively investigate the performance of their strategy. Without any surprise, familiarity with the system scored higher true positive scores than unfamiliarity. After quantifying those strategy, the count the occurrence of strategy based on the usage to find out the strategy that have the higher percentage of success. The six strategies were fixed for all the subjects which made the subjects to identify the design issues based on only those six strategies and no information on the accuracy of the hybrid strategies is not explained. Mixing up those strategies with each other to implement a hybrid strategy can yield some very interesting results. Also the order of using those strategies can affect the performance as well. However, their strategies can be used to identify design issue in code review discussions.

        
While it is very important to identify the design issues while the software is in the development phase, it is not practically feasible all the time to do that due to the extreme focus on achieving features and functionality within a specific amount of time. It is very important to automate the identification process of the design issues. Sousa in a more practical approach \cite{Sousa2018} has provided with a grounded theory on identifying design issues in source code level. For the experiment, projects are chosen with four specific criteria and developer are selected and categorized depending on their experience. The experiment has been divided in four activities: subjects characterization, training, problem identification and follow-up and six specific symptoms were investigated. Then the helpfulness of the system according to the developers is measured based on the count of the symptoms while it was applied and the number of contributions. Although some multi-trial industrial experiment to investigate the validation of their theory, some execution of empirical to access some more knowledge of the theory's proposition to successful detection.

The identification process and accuracy of trained individual can often differ from collaborative environment. Though everybody in the review team is familiar \cite{Sousa2017}, the identification can be different depending on very small factor like preference.    