% !TEX root = main.tex
\section{Presence}
\label{sect:presence}
Pull request based software development platform like Github provide developer a very effective communication medium to discuss about anything related to the project. Studies are being conducted by researchers to understand the discussions and categorize them. In a study in 2007, Zhi-Xing st al.,\cite{Li2007} tried to analyze the discussions in Github constructed a fine-grained taxonomy that organizes the discussions in 11 categories. They used discussions from Rails, ElasticSearch and Angular.js extracted by GHTorrent and from their own web crawler. From their manual classification, they found 7\% of them related to road-map mapping and 9\% of them is convention checking. Although these are not a very extensive list of design categories, it primarily demonstrate that, the developers talk about design in their discussions.

A very directed investigation on design was presented in \cite{Brunet2014a} by Brunet and co. where the authors specifically wanted to know if the developer discuss design. They considered full comment thread as a set of discussions and studied if one of the comments are related to the design of the system. Based on the manual classification of two unbiased authors and machine learning algorithms, they found the ratio between the total number of contributer and the number of developer that discuss design in a project. Following their study, they found that, at least 25 {$\pm$} 6\% discussions are somewhat related to design. They also found that, the developers who are talking about design are either the core developer team or has a significant amount of commits to the repository. This shows that, developer who has a deep understanding of the system cares about the design aspect of the system. Although Brunet\cite{Brunet2014a} provided a strong statement on the presence of design discussions, there is always some doubt of whether it may change and the design discussions rate may increase or decrease over time. Shakiba et al,.\cite{Shakiba2016} revisited the questions and provided some validation of the previous work by working with some diverse data from Github and SourceForge. From their manual classification they found that only 14\% of the discussions are related to design. Although, this studies provided some interesting insight of the design discussion, the category of the design discussions were not provided and thus they can not be organized properly.

Core team members often have to take some decision throughout the whole software lifecycle. These decision are often related to some design issues and the decisions that are made can have a significant effect on the system. Alkadhi et al.,\cite{Alkadhi2017} has termed this decisions as rationale and they have studied to find the rationale in the developer discussions in pull based software development. To achieve this, they have studied frequency of the rationale, completeness of the rationale and automatic extraction of the rationale. After coding 8702 messages with two individual coders, they came to the conclusion that, 9\% of the messages are about rationale. This study was extended by the same author in \cite{Alkadhi2018} investigated the developers who takes part in discussing rationale as well as the method in which they identify rationale and the accuracy of their method. When they expand their search, they found that, 25\% of the messages that they have analyzed contains some form of rationale and also it can vary significantly over project to project. Some large scale projects concentrating more on their architecture tend to have a large percentage of the discussion relate to rationale.

The studies mentioned so far talk about the presence of design in the developer discussions happening on different communication channel. But much is not analyzed about the form and content of the design discussions. Viviani et al.,\cite{Viviani2018a} did an in depth study of three pull request to demonstrate the structure of the design discussions. By analyzing the three pull requests, the authors have defined the elements and criteria of the design information that are present in written discussions. Their analysis paved the way to understand the argumentative structure of the discussions with the presence of questions and candidates. The study shows some reasons to support the acceptance and rejections as well as some counter-rejection to another rejection. Also it provide the design space that is present in the developer discussions. Following this work, the author extend this study to categorize and organize the form of design discussions in \cite{Viviani2018} where he tried to find out some specific topics related to design in the discussions. From the study of the discussions, he came up with a formal definition of the design point and sampled the discussions in paragraphs unlike \cite{Brunet2014a} and annotated a total of 10,790 paragraphs in which 2475 is related to design points. They categorized the design topics based on some Architecturally Significant Requirements and count the overall occurrence of each ASRs in terms of design. Although this is purely based on manual annotation by two coder and they investigated a small datasets, this study can be implemented in an extensive way to provide a developer easy access to design information.   

 

