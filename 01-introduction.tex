% !TEX root = main.tex
\section{Introduction}
\label{sect:introduction}
% the trend of pull request based software development
In this modern era of software development, pull based software development is being followed by every project that uses shared repository approach \cite{Gousios2014}. The standard way to achieve this approach is the code review process. During this code review process, necessary changes, feedback and reviews are given under a specific pull request in the form of comments. These comments often express important information on how are the stakeholders influencing the outcome or what are the roles of power relationships in evaluating contributions and how to develop software requirements through discussion \cite{Tsay2014}. The discussions also provide insights on the project convention and architecture. These discussions can also act as the prominent driving agent for any decision that can affect the architecture of the software.

% what is design discussion 
There are many resources that define design in terms of a software. But when it comes to identify design discussion, it varies according to the context of the discussion. A design point \cite{Viviani2018} can be a discussion that has the relation or impact on the decision of a software system's design that a software development team is required to take. Discussions on the non functional requirements that has some effects on the overall project structure and design can also be considered as design discussions \cite{Sousa2018}. Discussions that contains implementation issues, future plans, OS support, code standard, test ability, robustness, performance, runtime optimization, configuration files, flags, option and documentation can be considered as a design discussion \cite{Viviani2018}.

% problem of not having design 
Life of a newcomer in a new project specifically in an open source project can be very hard without any proper architectural and design view of the software system. \cite{Steinmacher2014}. A study by Martin \cite{Robillard2009} shows that design is a sub category that appear multiple times in categories like Resources and Structure is an obstacle to learn APIs. There may be some design discussion that is very helpful for the newcomers. But those discussions are lost due to the lack of documentation and organization. On the other hand, even if it is good for an open source project to have as many contributors as possible, giving answers to the design related questions can be also very time consuming and redundant for the author of the project. The author has to answer the same kinds of questions or may need to get involved in the same kinds of discussions over and over again. Many software development tasks can be simplified and more directed if it was possible to identify those discussions that has design points in it and document them is a structured way for the future use. such innovation can be in turn reduce the time that goes into reviewing a pull request.

\begin{table}
	\caption{Paragraph that contains design information.\cite{Viviani2018}}	
		\textbf{\textit{``leaky abstraction in the sense that your abstraction is saying too much about the implementation -- you 're declaring to the world that you had to make compromises on your API to get other outcomes (performance), there has to be a tradeoff between pure perf and the best internal implementation and the API we expose to users and I'm here representing the API and this is that tradeoff discussion''}}	
\end{table}

\begin{table}
	\caption{Paragraph not related to design.\cite{Viviani2018}}
	\textbf{\textit{``Ok. I was pretty keen on getting 1.7.2 within a week or so with a fix to a shared build. Guessing 2.0 might make that easier since we'd branch off to master/1.x/2.x?''}}
\end{table}

% Is it actually happening
We can get the motivation to survey the presence of design discussion that happens in online developer community from the fact that, design appears in various literature as important as achieving other functionality and features in a software system. Yet, it is very difficult to find any ongoing design related work in the projects \cite{Brunet2014a}. Although developers do not develop design related documentation most often, discussions on design issues happens in various communication channel within the developer community. Studies shows that, 25\% of the discussions in a project are related to design points and almost 26\% of the developers contribute to at least one design discussion \cite{Brunet2014a}. Another study shows that, an average of 22\% of the discussions are about design point with a high of 24\% and low of 20\% \cite{Viviani2018}. Design issues has some significant impact on the source code itself. To detect the design issues in the source code level, several academic and industrial tool has already been proposed \cite{Sousa2018}. Also there are some proposed method to recommend supervisory assistance in performing software changes \cite{Kagdi2008}. There is also one study that demonstrate the effect of architectural and design choices on different behavior of technical debt that leads to rejection of code contribution \cite{Curtis2012a}.

% Future works
There are many practical implications of mining the design discussion in the code review process. We can use those design points to generate an extractive summery of the software system that can be used as a form of documentation following the step shown by Rodeghero in \cite{Rod2017}. If the design topics that are being discussed during a code review process, we can invite the project member who has expertise on reviewing design related issues. A ranked list approach \cite{Kagdi2008} to measure the expertise of developer in the open source projects. We can develop tools that can be used to visualize the effect of particular design discussion in terms of manipulating any decision or making any change in the code \cite{Viviani2018a}. Some evaluation tools can be develop to quantify the architectural and design trade offs a developer has to make attain certain feature. This work can be extended further to implement some kind of recommender system or even some self adaptive auto-correct system to fix design issues.

% thesis 
Manual labeling of a pull request comment thread exits and one of the authors or core developers has to manually label each pull request to keep track of the topic that they need to address. It would be much more efficient for the maintainer of the projects if this labeling can be done automatically. Also for not keeping up with the recent trend or just simply by mistake, the maintainers can give incorrect feedback or reviews that can lead to rejection of a potential effective contribution. As Buschmann \cite{Buschmann2007} suggest that pattern and design based software engineering will be the future trend of software development, after identifying design point in the discussions, can we develop a bot that can detect design flaws in code review and possibly give some suggestions based on it's knowledge base?        


   
